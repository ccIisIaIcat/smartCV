\documentclass{article}
\usepackage{xeCJK}
\usepackage{hyperref}
\usepackage{enumitem}

% 设置中文字体
\setCJKmainfont{SimSun}[AutoFakeBold=true]
\setCJKsansfont{SimHei}
\setCJKmonofont{SimFang}

% 设置超链接样式
\hypersetup{
    colorlinks=true,
    linkcolor=blue,
    filecolor=blue,
    urlcolor=cyan,
    pdftitle={SmartCV - 智能简历生成系统},
    pdfauthor={SmartCV}
}

\begin{document}
\title{SmartCV - 智能简历生成系统}
\author{生成建立后保存pdf在右上角}
\date{\today}

\maketitle

\section{项目简介}
SmartCV 是一个基于 Go+Vue 开发的智能简历免费小应用,它能够帮助用户快速生成匹配招聘需求的 LaTeX 格式简历。系统使用 OpenAI 的 国内镜像closeAI(奇怪的起名方式)来生成和修改简历内容。\newline
建议访问我的git本地部署:\href{https://github.com/ccIisIaIcat}{我的GitHub主页}\newline
由于不收费,所以需要自己提供api key \href{https://referer.shadowai.xyz/r/1026582}{CloseAI:https://referer.shadowai.xyz/r/1026582}该链接可提供优惠额度, \newline

\section{主要功能}
\begin{itemize}
    \item 智能简历生成:根据用户输入的个人介绍和岗位需求自动生成专业简历
    \item 实时预览:支持 PDF 实时预览功能
    \item 内容优化:根据你想撒多大的谎来设置(诚实、适中、积极、创意、专家)
    \item 灵活编辑:支持中文 LaTeX 源码直接编辑和实时编译
\end{itemize}

\section{使用说明}
\subsection{准备工作}
在使用本系统之前,您需要:
\begin{enumerate}
    \item 准备 OpenAI国内镜像 API Key(可以通过 \href{https://referer.shadowai.xyz/r/1026582}{CloseAI} 获取优惠额度的 API Key)
    \item 准备个人基本信息和工作经历
    \item 确定目标职位的要求
\end{enumerate}

\subsection{使用步骤}
\begin{enumerate}
    \item 输入 API Key:在系统顶部输入框中填入您的 OpenAI国内镜像(CloseAI) API Key
    \item 选择优化级别:根据需求选择简历内容的优化程度
    \item 填写个人介绍:在文本框中输入您的个人经历和技能特长
    \item 填写职位要求:输入目标职位的具体要求
    \item 生成简历:点击"生成简历"按钮,系统将自动生成 LaTeX 格式的简历
    \item latex直接修改:
    \begin{itemize}
        \item 在右侧预览区查看生成的 PDF 效果
        \item 如需修改,可以直接编辑 LaTeX 代码或在修改建议框中描述需要调整的内容
        \item 点击编译进行在线编译
    \end{itemize}
    \item 自然语言修改:
    \begin{itemize}
        \item 在右侧预览区查看生成的 PDF 效果
        \item 如需修改,可以用大白话描述需要调整的内容(文本格式都可以)
        \item 点击"更新简历"进行内容优化
    \end{itemize}
\end{enumerate}

\section{注意事项}
\begin{itemize}
    \item API Key 安全:请妥善保管您的 API Key,不要泄露给他人,本网站不会进行任何记录
    \item 内容真实:记得过高等级的优化级别可能伴随更多谎言
    \item 定期更新:建议定期更新简历内容,保持信息的时效性
    \item 格式调整:如果对生成的格式不满意,可以直接修改 LaTeX 源码
\end{itemize}

\section{技术支持}
如果在使用过程中遇到问题,可以:
\begin{itemize}
    \item 查看系统提供的错误提示信息
    \item 检查 API Key 是否正确
    \item 确认网络连接是否正常
    \item 尝试刷新页面重新操作
\end{itemize}

\end{document}